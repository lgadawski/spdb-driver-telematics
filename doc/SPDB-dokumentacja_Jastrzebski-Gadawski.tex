\documentclass[]{scrartcl}
\usepackage{amsmath}
\usepackage{polski}
\usepackage[utf8]{inputenc}
\usepackage{graphicx}
\usepackage{url}

%opening
\title{Rozpoznawanie kierowców na podstawie profilu jazdy}
\author{Jastrzębski Mateusz, Gadawski Łukasz}

\begin{document}

\maketitle

\begin{abstract}
Abstract if need
\end{abstract}

\section{Treść zadania}
Realizacja projektu polega na stworzeniu rozwiązania zadania konkursowego dotyczącego rozpoznawania kierowców na podstawie podanych danych. Organizator konkursu dostarcza profile jazdy dla 3613 kierowców. Dla każdego kierowcy dane są 200 plików CSV opisujących pojedynczą podróż. Każdy taki plik zawiera współrzędne kierowcy na płaszczyźnie w każdej sekundzie jego podróży, np:
\begin{verbatim}
x,y
0.0,0.0
18.6,-11.1
36.1,-21.9
...
\end{verbatim}

W każdym zbiorze 200 podróży zamieszczono kilka losowych tras innych kierowców (niekoniecznie z dostarczonego zbioru).

Zadaniem projektu jest identyfikacja podróży, które nie należą do danego kierowcy. Dane wynikowe powinny zostać przedstawione jako tablica wartości prawdopodobieństwa, że konkretna ścieżka należy do aktualnego kierowcy, np:
\begin{verbatim}
driver_trip,prob
1_1,0.8
1_2,0.4
1_3,0.7
1_4,0.9
...
\end{verbatim}

\section{Wykorzystane algorytmy}
\subsection{Random Forest}
Celem metod zespołowych (ang. \textit{ensemble methods}) jest połączenie prognozy estymatorów wraz z odpowiednim algorytmem uczącym, aby uzyskać możliwość oceny pojedynczej próbki w jakim stopniu jest zgodna z innymi próbkami na podstawie wspomnianych estymatorów. Do tego celu wyróżnia się dwa rodzaje metod 
%[http://scikit-learn.org/stable/modules/ensemble.html#parameters]
:
\begin{itemize}
	\item metody uśrednianiające (ang. \textit{averaging methods}) - polegające na stworzeniu estymatorów niezależnie, a następnie na uśrednieniu ich prognoz, np. Random Forest,
	\item metody wzmacniające (ang. \textit{boosting methods}) - polegające na sekwencyjnej budowie estymatorów 
	
\end{itemize}

Jednym z wykorzystanych algorytm jest algorytm Random Forest [cite]. W projekcie została wykorzystana implementacja w języku python z pakietu scikit-learn [cite]. Jego praktyczne wykorzystanie wiąże się z następującymi krokami:
\begin{enumerate}
	\item inicjalizacja modelu,
	\item dostarczenie danych,
	\item predykcja nowych.
\end{enumerate}
\end{document}
